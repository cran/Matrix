\hypertarget{dtCMatrix_8c}{
\subsection{dt\-CMatrix.c File Reference}
\label{dtCMatrix_8c}\index{dtCMatrix.c@{dtCMatrix.c}}
}
{\tt \#include \char`\"{}dt\-CMatrix.h\char`\"{}}\par
\subsubsection*{Functions}
\begin{CompactItemize}
\item 
SEXP \hyperlink{dtCMatrix_8c_a0}{tsc\_\-validate} (SEXP x)
\item 
SEXP \hyperlink{dtCMatrix_8c_a1}{tsc\_\-transpose} (SEXP x)
\item 
SEXP \hyperlink{dtCMatrix_8c_a2}{tsc\_\-to\_\-dg\-TMatrix} (SEXP x)
\item 
int \hyperlink{dtCMatrix_8c_a3}{parent\_\-inv\_\-ap} (int n, int count\-Diag, const int pr\mbox{[}$\,$\mbox{]}, int ap\mbox{[}$\,$\mbox{]})
\item 
void \hyperlink{dtCMatrix_8c_a4}{parent\_\-inv\_\-ai} (int n, int count\-Diag, const int pr\mbox{[}$\,$\mbox{]}, int ai\mbox{[}$\,$\mbox{]})
\item 
SEXP \hyperlink{dtCMatrix_8c_a5}{Parent\_\-inverse} (SEXP par, SEXP unitdiag)
\end{CompactItemize}


\subsubsection{Function Documentation}
\hypertarget{dtCMatrix_8c_a4}{
\index{dtCMatrix.c@{dt\-CMatrix.c}!parent_inv_ai@{parent\_\-inv\_\-ai}}
\index{parent_inv_ai@{parent\_\-inv\_\-ai}!dtCMatrix.c@{dt\-CMatrix.c}}
\paragraph[parent\_\-inv\_\-ai]{\setlength{\rightskip}{0pt plus 5cm}void parent\_\-inv\_\-ai (int {\em n}, int {\em count\-Diag}, const int {\em pr}\mbox{[}$\,$\mbox{]}, int {\em ai}\mbox{[}$\,$\mbox{]})}\hfill}
\label{dtCMatrix_8c_a4}


Derive the row index array for the inverse of L from the parent array

\begin{Desc}
\item[Parameters:]
\begin{description}
\item[{\em n}]length of parent array \item[{\em count\-Diag}]0 for a unit triangular matrix with implicit diagonal, otherwise 1 \item[{\em pr}]parent vector describing the elimination tree \item[{\em ai}]row index vector of length ap\mbox{[}n\mbox{]} \end{description}
\end{Desc}
\hypertarget{dtCMatrix_8c_a3}{
\index{dtCMatrix.c@{dt\-CMatrix.c}!parent_inv_ap@{parent\_\-inv\_\-ap}}
\index{parent_inv_ap@{parent\_\-inv\_\-ap}!dtCMatrix.c@{dt\-CMatrix.c}}
\paragraph[parent\_\-inv\_\-ap]{\setlength{\rightskip}{0pt plus 5cm}int parent\_\-inv\_\-ap (int {\em n}, int {\em count\-Diag}, const int {\em pr}\mbox{[}$\,$\mbox{]}, int {\em ap}\mbox{[}$\,$\mbox{]})}\hfill}
\label{dtCMatrix_8c_a3}


Derive the column pointer vector for the inverse of L from the parent array

\begin{Desc}
\item[Parameters:]
\begin{description}
\item[{\em n}]length of parent array \item[{\em count\-Diag}]0 for a unit triangular matrix with implicit diagonal, otherwise 1 \item[{\em pr}]parent vector describing the elimination tree \item[{\em ap}]array of length n+1 to be filled with the column pointers\end{description}
\end{Desc}
\begin{Desc}
\item[Returns:]the number of non-zero entries (ap\mbox{[}n\mbox{]}) \end{Desc}
\hypertarget{dtCMatrix_8c_a5}{
\index{dtCMatrix.c@{dt\-CMatrix.c}!Parent_inverse@{Parent\_\-inverse}}
\index{Parent_inverse@{Parent\_\-inverse}!dtCMatrix.c@{dt\-CMatrix.c}}
\paragraph[Parent\_\-inverse]{\setlength{\rightskip}{0pt plus 5cm}SEXP Parent\_\-inverse (SEXP {\em par}, SEXP {\em unitdiag})}\hfill}
\label{dtCMatrix_8c_a5}


\hypertarget{dtCMatrix_8c_a2}{
\index{dtCMatrix.c@{dt\-CMatrix.c}!tsc_to_dgTMatrix@{tsc\_\-to\_\-dgTMatrix}}
\index{tsc_to_dgTMatrix@{tsc\_\-to\_\-dgTMatrix}!dtCMatrix.c@{dt\-CMatrix.c}}
\paragraph[tsc\_\-to\_\-dgTMatrix]{\setlength{\rightskip}{0pt plus 5cm}SEXP tsc\_\-to\_\-dg\-TMatrix (SEXP {\em x})}\hfill}
\label{dtCMatrix_8c_a2}


\hypertarget{dtCMatrix_8c_a1}{
\index{dtCMatrix.c@{dt\-CMatrix.c}!tsc_transpose@{tsc\_\-transpose}}
\index{tsc_transpose@{tsc\_\-transpose}!dtCMatrix.c@{dt\-CMatrix.c}}
\paragraph[tsc\_\-transpose]{\setlength{\rightskip}{0pt plus 5cm}SEXP tsc\_\-transpose (SEXP {\em x})}\hfill}
\label{dtCMatrix_8c_a1}


\hypertarget{dtCMatrix_8c_a0}{
\index{dtCMatrix.c@{dt\-CMatrix.c}!tsc_validate@{tsc\_\-validate}}
\index{tsc_validate@{tsc\_\-validate}!dtCMatrix.c@{dt\-CMatrix.c}}
\paragraph[tsc\_\-validate]{\setlength{\rightskip}{0pt plus 5cm}SEXP tsc\_\-validate (SEXP {\em x})}\hfill}
\label{dtCMatrix_8c_a0}


